\section{Introduction to Compiler Design}

This tutorial section is focussed to know about the basics of Compiler blocks.


\begin{question}
	 Compilers are broken into several chunks called passes that communicate with one another via temporary files. Justify the construction of compiler as a complex program.
\end{question}

\begin{question}
	Define Lexer. Describe the role of lexical analysis in compiler construction.
\end{question}

\begin{question}
	Define parsing. Compare the parsing process of English sentence: `Ram sees Laxman flee` with the parsing process of Expression (A * B - C * D).
\end{question}

\begin{question}
	What is Code Generation in compilation process? Discuss on the basic principles.
\end{question}

\begin{question}
	Write short notes on:
  \begin{enumerate}
    \item Context Free Grammar
    \item Backus-Naur Form
    \item Syntax Diagram
  \end{enumerate}
\end{question}

\begin{question}
Write a grammar that recognizes a C variable declaration made up of the following keywords:
\begin{verbatim}
 int  char long float double signed unsigned short const volatile
\end{verbatim}
and a variable name.
\end{question}

\begin{question}

 Write a grammar that recognizes a C variable declaration made up only of legal
combinations of the following keywords:
\begin{verbatim}
    int char long float double signed unsigned short const volatile
\end{verbatim}
and a variable name. The grammar should be able to accept all such legal declarations. For example, all the following should be accepted:
\begin{verbatim}
  volatile unsigned long int x;
  unsigned long volatile int x;
  long unsigned volatile int x;
  long volatile unsigned int x;
\end{verbatim}

but something like this should not be accepted:
\begin{verbatim}
  unsigned signed short long x;
\end{verbatim}
\end{question}

\begin{question}

Write a grammar (and a recursive-descent compiler for that grammar) that translates an English description of a C variable into a C-style variable declaration. For example, the input:
\begin{verbatim}
  x is a pointer to an array of 10 pointers to functions that return int.
  y is an array of 10 floats.
  z is a pointer to a struct of type a struct.
\end{verbatim}
should be translated to:
\begin{verbatim}
  int (* (*x) [10]) ();
  float y[10];
  struct a_struct *z;
\end{verbatim}
\end{question}

% \includepdf[pages={1}]{assignments/DataTypesSummary.pdf}
